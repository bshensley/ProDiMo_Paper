\documentclass[12pt,preprint]{aastex}
\usepackage{amsmath}
\usepackage{amssymb}
\usepackage{graphicx}
\usepackage{subfigure}
\usepackage{color} % For use only with the \remark{} macro below.

\slugcomment{21 May, 2012}

\newcommand{\unit}[1]{\,{\rm #1}}
\newcommand{\bfnabla}{{\mbox{\boldmath $\nabla$}}}
\newcommand{\gcmsq}{\unit{g\,cm^{-2}}}
\newcommand{\kms}{\,\mbox{km s}^{-1}}
\newcommand{\kpc}{\,\mbox{kpc}}
\newcommand{\pc}{\,\mbox{pc}}
\newcommand{\QT}{Q_{\rm T}}
\newcommand{\RH}{R_{\rm H}}
\newcommand{\rS}{r_{\rm S}}
\newcommand{\yr}{\,\mbox{yr}}
\newcommand{\Gyr}{\,\mbox{Gyr}}
\newcommand{\au}{\,\textsc{au}}
\newcommand{\msun}{M_\odot}
\newcommand{\mbh}{M_{\rm{BH}}}
\newcommand{\del}{\partial}
\renewcommand{\bv}{{\mbox{\boldmath $v$}}}
\newcommand{\avg}[1]{\langle\langle#1\rangle\rangle}
%\newcommand{\<}{\,\langle\langle}
%\renewcommand{\>}{\,\rangle\rangle}
\newcommand{\solaryr}{M_{\odot}\ \mbox{yr}^{-1}}
\newcommand{\prodimo}{{\sc ProDiMo}\,}
\newcommand{\remark}[1]{{\color{red}\bf#1}} % Macro for editorial questions in boldface

\shortauthors{Brandon Hensley \& Jeremy Goodman} \shorttitle{Signatures of accretion}

\begin{document}

\title{Signatures of accretion in protostellar disks}

\author{Brandon Hensley\altaffilmark{1}  \& Jeremy Goodman\altaffilmark{1}  }
\affil{$^1$Department of Astrophysical Sciences, Princeton
University, Princeton, NJ 08544, USA}
\begin{abstract}
To be written
\end{abstract}

\keywords{ accretion, accretion disks --- stars: formation }

\section{Introduction}
\label{sec:intro}
For many years, observational constraints on dynamical models of protostellar disks were
limited to broad-band spectral-energy distributions (SEDs), accretion rates as determined
from excess emission on the blue side of the stellar photospheric emission, and inferences
concerning the primordial solar nebula based on the present solar system.  The recent
harvest of infrared molecular spectra from observatories such as {\it Spitzer}, {\it
  Herschel}, Keck, and the VLT promises by the sheer number of bits involved to provide
much more detailed constraints to modelers, if only one can interpret these new data
intelligently \remark{References here}.

A longstanding dynamical question is the nature of the mechanism responsible for
extracting angular momentum from the accreting material.  The main suspect is turbulence
driven by the magnetorotational instability (MRI), but a difficulty with this mechanism is
the very low ionization fraction of the disk gas, so that coupling to magnetic fields may
be insufficient for MRI \remark{many references here}.  At distances of a few tenths to a
few $\au$ from the accreting low-mass protostar, it is believed that stellar X-rays and
other nonthermal sources of ionization support MRI only in near-surface layers of the disk
having typical column densities at $1\au$ of order $10\gcmsq$, much less than the total
disk column\citep[and references
therein]{Gammie96,Glassgold+etal97,Bai+Goodman2009}.

\subsection{Energetics}\label{subsec:energetics}

As is well known, a razor-thin disk intercepts one quarter of the stellar luminosity, and
The stellar flux impinging on one side of the disk at radius $r\gg R_*$ is $F_*(r) \approx L_* R_*/6\pi r^3$,
$R_*$ and $L_*$  being the photospheric radius and luminosity.  The
correction $2R_*/3r$  to the inverse-square law accounts for the grazing incidence of the
rays.  On the other hand, a disk accreting steadily at rate $\dot M$ radiates $F_{\rm acc}
= 3GM\dot M/8\pi r^3$ in addition to any reprocessed stellar light.  Since both fluxes
scale as $r^{-3}$, there is a characteristic mass accretion rate at which the heat of
accretion equals the irradiation.  This is
\begin{equation}
  \label{eq:Mdoteq}
\dot M_{\rm eq} = \frac{4}{9\pi}\frac{R_* L_*}{GM_*} \approx 4.5\times 10^{-9} 
\left(\frac{L_*}{L_\sun}\right) \left(\frac{R_*}{R_\sun}\right)
\left(\frac{M_\sun}{M_*}\right) \,{\rm \dot M_\sun\,yr}^{-1}\,.
\end{equation}
The typical accretion rate for classical T+Tauri stars is $\sim 10^{-8}\,{\rm \dot
  M_\sun\,yr}^{-1}$, though with wide scatter and with strong dependence on the
protostellar mass, $\dot M\propto M_*^2$ \citep{Hartmann+Calvet+Gullbring+etal1998,
  Muzerolle+Luhman+Briceno2005,Fang+vanBoekel+Wang+etal2009}.  That eq.~\eqref{eq:Mdoteq}
should predict observed accretion rates is probably no coincidence.  By selection, T~Tauri
have not yet reached the main sequence, so that their luminosities derive from
gravitational contraction.  It is therefore natural that their accretion timescales
$M_*/\dot M$ should be comparable to their Kelvin-Helmholtz times.

However, protostellar disks are probably not flat but rather flaring, i.e. the vertical
density scale height ($H$) increases faster than linearly with radius, and a flaring disk
intercepts more of the stellar light than a flat disk would. Consequently, beyond a few
stellar radii in the disk, the accretion power is locally rather small compared to the
reprocessed emission.  To illustrate this, the effective temperature corresponding to an
accretion rate of $10^{-8}\,{\rm \dot M\,yr^{-1}}$ onto a solar-mass star is only
$85\,r_{\au}^{-3/4}\,{\rm K}$.  Infrared spectral-energy distributions clearly require
higher temperatures than this, as can be achieved by a self-consistently flaring,
passively reprocessing disk \citep{Chiang+Goldreich1997}.   Only in the boundary layer
between the disk and the star should the accretion strongly outshine
the reprocessed light, which is why boundary-layer emission is used to measure
accretion rates.

As already noted, it would be useful to find emission that could be directly associated
with accretion at larger radii, particularly at $r\sim 1\au$ where MRI is most
problematic.  The net frequency-integrated emission coming from these larger radii is not
very useful as an accretion diagnostic because of the dominant reprocessed
light.\footnote{except when the accretion rate temporarily rises far above
  $10^{-8}\,{M_\sun\,yr^{-1}}$, as may be the case in FU~Orionis systems
  \remark{references here}.}  However, molecular lines observed in some systems appear to
require that some fraction of the gas at $r\lesssim 1\au$ is much hotter than would occur
in a passively reprocessing disk if it radiated locally as a black body
\citep{Salyk_etal08,Carr_Najita08}.  The emitting gas is probably many density scale
heights from the disk midplane, where the density is relatively low and cooling
inefficient.  To model the hot gas properly, one has to consider individual heating,
cooling, and line-forming processes in some detail; the assumption of local thermodynamic
equilibrium (LTE) is inadequate.  One finds that the main part of the stellar light, with
a color temperature $\sim 4000\mbox{--}5000\,{\rm K}$, cannot explain the high gas
temperature.  Instead, in a passively reprocessing model, the hot gas depends upon
far-ultraviolet (and perhaps also X-ray) excesses in the stellar spectrum.  Because
T~Tauri stars are chromospherically active, these excesses are strong but typically still
less than one percent of the bolometric luminosity.  Therefore, if the accretion process
can efficiently heat the low-density, high-altitude gas, it might produce a signature in
the molecular emissions that could be distinguished from purely passive reprocessing.
This has been the motivation for our study.

\subsection{Plan of the paper.}

We use the \prodimo code to model the microphysics of the disk gas, including non-LTE
level populations, gas-dust interactions, radiative transitions, and radiative transfer.
The features of this code are well documented by its authors \citep{Woitke+etal2009,
Kamp+Tilling+Woitke+Thi2010,Woitke+Riaz+Duchene2011}, though it is still actively developing.
\prodimo comes ``out of the box'' with a simple prescription for accretion heating, but
this is characterized by a constant effective viscosity, corresponding to a constant
heating rate per unit mass (dependent upon shear rate but independent
of density and altitude) \remark{They have since added a different
  treatment for the viscous heating by assuming an alpha parameter,
  computing the dissipation at a given radius based on Mdot, then
  distributing the heating in altitude as $\rho^2$}.

% The temperature of the
% emitting molecules is estimated from the exitation energies of the upper states of the
% transitions involved, while the radius is constrained by the line widths, where measured,
% on the assumption that these represent orbital motion.


\section{Disk Modeling with ProDiMo}
\label{sec:prodimo}

\subsection{ProDiMo}

Insert a description of the code, the knobs tweaked, and extra modules
(notably Xray) here.

\subsection{Disk Structure}
The structure of a disk is determined in ProDiMo entirely through five
parameters: the mass of the disk $M_{\mathrm{disk}}$, the mass of the
star $M_\star$, the inner radius of the disk $R_{\mathrm{in}}$, the
outer radius of the disk $R_{\mathrm{out}}$, and the radial power law
index of the column density $\epsilon$ as defined by:

\begin{equation}
\ \Sigma\left(r\right) = \Sigma_0 r^{-\epsilon} .
\end{equation} 

We emulate the MMSN as described by Hayashi \remark{add ref} by taking
$\Sigma_0 =1700\gcmsq$ and $\epsilon = 1.5$. If
we set $R_\mathrm{in}  = 0.25\,\au$ and $R_\mathrm{out} = 200\,\au$,
consistent with previous ProDiMo modeling of TW Hya \remark{citation?},
then we can infer the mass of the disk by the relation

\begin{equation}
\ M_{\mathrm{disk}} = 2\pi \int_{R_{\mathrm{in}}}^{R_\mathrm{out}} \!
\Sigma\left(r\right) r \mathrm{d}r ,
\end{equation}
which yields $M_\mathrm{disk} = 0.033 M_\odot$. 

Finally, we follow \citet{Bai+Goodman2009} and take $M_\star = 1 M_\odot$. Hence, all five
necessary structural parameters have been determined.

\section{Simulation Results}

In this section, we present a suite of ProDiMo simulations that probe
the effects of different heating mechanisms on the molecular line
emission of the disk. 

\subsection{Baseline Model}

Outline
\begin{itemize}
\item Dust abundance, why baseline of $10^{-4}$
\item Discuss assumptions on chemistry, what is included, what is not
  (may go in ProDiMo section above, but needs written either way)
\item Discuss X-ray luminosity (should ``baseline model'' have any
  X-ray at all?)
\end{itemize}

\remark{Make table with ProDiMo parameters}

\remark{Make grid of standard disk structure plots}

\subsection{Viscous Heating Models}

Outline
\begin{itemize}
\item Discuss HT heating model and its implementation in ProDiMo
\item Discuss effects on disk temperature
\item Discuss effects on SED (continuum + line emission), showing ratio plots with
  baseline model
\item Emphasize differences that could be observable
\end{itemize}

\subsection{X-Ray Heating}

Outline
\begin{itemize}
\item As above, but with emphasis on contrasts with viscous heating
  model, esp. observables
\end{itemize}

\subsection{UV Heating}

\subsection{Effects of Dust Abundance and Settling}

\section{Conclusion}


\bibliographystyle{apj}
\bibliography{PDpaper}

\end{document}
